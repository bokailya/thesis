\subsection{Метод граничного управления}

Рассмотрим обратную задачу для волнового уравнения.

\begin{align*}
    \frac{\partial^2 u}{\partial t^2}
    &=
    c^2
    \left(
    \frac{\partial^2 u}{\partial x_1^2}
    + \frac{\partial^2 u}{\partial x_2^2}
    + \cdots
    + \frac{\partial^2 u}{\partial x_n^2}
    \right)
    ,
    t \in \left( 0, T \right)
    \\
    u \left( \mathbf{x}, 0 \right) &= 0 \\
    \frac{\partial u}{\partial t} \left( \mathbf{x}, 0 \right) &= 0 \\
    u \left( \mathbf{x}, t \right)
    &=
    \phi \left( \mathbf{x}, t \right),
    t \ \in \left( 0, T \right), x \in \partial \Omega
    \\
    \frac{\partial u}{\partial \mathbf{n}} \left( \mathbf{x}, t \right)
    &=
    f \left( \mathbf{x} \right)
    ,
    x \in \partial \Omega
\end{align*}

Обратная задача граничного управления состоит в
отыскании функции \( f \left( \mathbf{x} \right) \),
зная функию \( \phi \left( \mathbf{x} \right) \).
