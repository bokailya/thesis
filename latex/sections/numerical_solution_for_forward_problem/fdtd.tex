\subsection{Finite-Difference Time-Domain}

Рассмотрим задачу Коши для волнового уравнения
с однородным граничным условияем Неймана.

\begin{align*}
    \frac{\partial^2 u}{\partial t^2}
    &=
    c^2
    \left(
    \frac{\partial^2 u}{\partial x_1^2}
    + \frac{\partial^2 u}{\partial x_2^2}
    + \cdots
    + \frac{\partial^2 u}{\partial x_n^2}
    \right)
    ,
    t \in \left( 0, T \right)
    \\
    u \left( \mathbf{x}, 0 \right) &= u_0 \left( \mathbf{x} \right) \\
    \frac{\partial u}{\partial t} \left( \mathbf{x}, 0 \right) &= 0 \\
    \frac{\partial u}{\partial \mathbf{n}} \left( \mathbf{x}, t \right) &= 0
    ,
    x \in \partial \Omega
\end{align*}

Численное решение задачи можно получить
с использованием метода Finite-Difference Time-Domain
\cite{understanding_the_finite-difference_time-domain_method}.

Акустические волны есть изменения давления
\( u \left( \mathbf{x}, t \right) \).

\( c \left( \mathbf{x} \right) \)
- скорость звука в точке \( \mathbf{x} \) (не зависит от времени \( t \)).

Из физических соображений обычно рассматривают случаи \( n \le 3 \).

Рассмотрим случай \( n = 3 \).

\begin{align*}
    \frac{\partial^2 u}{\partial t^2}
    &=
    c^2
    \left(
        \frac{\partial^2 u}{\partial x^2}
        + \frac{\partial^2 u}{\partial y^2}
        + \frac{\partial^2 u}{\partial z^2}
    \right)
    ,
    t \in \left( 0, T \right)
    \\
    u \left( \mathbf{x}, 0 \right) &= u_0 \left( \mathbf{x} \right) \\
    \frac{\partial u}{\partial t} \left( \mathbf{x}, 0 \right) &= 0 \\
    \frac{\partial u}{\partial \mathbf{n}} \left( \mathbf{x} , t \right) &= 0
    ,
    x \in \partial \Omega
    \\
\end{align*}

Пусть \( \rho \left( \mathbf{x} \right) = \lim \frac{m}{V} \)
- плотность в точке \( \mathbf{x} \) (не зависит от времени \( t \)).


Пусть \( \Delta x_i = x_i - x_i^0 \).

Рассмотрим куб со стороной \( \abs{\Delta x_i} \).

Площадь поперечного сечения
\( A \left( \Delta x_i \right) = \left( \Delta x_i \right) ^2 \).

Объём \( V \left( \Delta x_i \right) = A \abs{\Delta x_i} \).

Масса \( m \left( \Delta x_i \right) \).

\setlength{\unitlength}{1cm}
\begin{picture}(8, 5)
    % Figure 1
    \put(2, 2){\line(0, 1){2}}
    \put(2, 2){\line(1, 0){2}}
    \put(2, 4){\line(1, 0){2}}
    \put(4, 2){\line(0, 1){2}}
    \put(2, 1.5){\( x_i^0 \)}
    \put(4, 1.5){\( x_i \)}
    \put(2, 3){\vector(1, 0){0.5}}
    \put(4, 3){\vector(-1, 0){0.5}}
    \put(2.2, 3.2){\( F_0 \)}
    \put(3.5, 3.2){\( F_1 \)}
    \put(1.5, 3.5){\( A \)}
    \put(4.2, 3.5){\( A \)}
    \put(0, 0){\vector(1, 0){5}}
    \put(1, 0){\circle*{0.1}}
    \put(4.5, 0.3){\( x_i \)}

    % Figure 2
    \put(8, 2){\line(0, 1){2}}
    \put(8, 2){\line(1, 0){2}}
    \put(8, 4){\line(1, 0){2}}
    \put(10, 2){\line(0, 1){2}}
    \put(8, 1.5){\( x_i \)}
    \put(10, 1.5){\( x_i^0 \)}
    \put(8, 3){\vector(1, 0){0.5}}
    \put(10, 3){\vector(-1, 0){0.5}}
    \put(8.2, 3.2){\( F_1 \)}
    \put(9.5, 3.2){\( F_0 \)}
    \put(7.5, 3.5){\( A \)}
    \put(10.2, 3.5){\( A \)}
    \put(6, 0){\vector(1, 0){5}}
    \put(7, 0){\circle*{0.1}}
    \put(10.5, 0.3){\( x_i \)}
\end{picture}

Проекция сил действующих на куб на ось \( Ox_i \): 

\begin{align*}
    F_{x_i} \left( \Delta x_i \right)
    &= \sign \left( \Delta x_i \right) \left( F_0 + F_1 \right)
    \\
    &=
    \sign \left( \Delta x_i \right)
    \left(
    u \left( \mathbf{x} \right) A
    - u \left( \mathbf{x} + \Delta x_i \mathbf{e}_i \right) A
    \right)
    \\
    &=
    \sign \left( \Delta x_i \right)
    \left(
    u \left( \mathbf{x} \right)
    - u \left( \mathbf{x} + \Delta x_i \mathbf{e}_i \right)
    \right) A
\end{align*}

где \( \mathbf{e}_i \) - базисный вектор на оси \( Ox_i \).

Второй закон Ньютона:

\begin{align*}
    \mathbf{F} &= m \mathbf{a} \\
    m \mathbf{a} &= \mathbf{F} \\
    m a_{x_i} &= F_{x_i} \\
    m \frac{\partial v_{x_i}}{\partial t}
    &=
    \sign \left( \Delta x_i \right)
    \left(
    u \left( \mathbf{x} \right)
    - u \left( \mathbf{x} + \Delta x_i \mathbf{e}_i \right)
    \right)
    A
    \\
    \frac{m}{V} \frac{\partial v{x_i}}{\partial t}
    &=
    \sign \left( \Delta x_i \right)
    \left(
    u \left( \mathbf{x} \right)
    - u \left( \mathbf{x} + \Delta x_i \mathbf{e}_i \right)
    \right)
    \frac{A}{V}
    \\
    \frac{m}{V} \frac{\partial v_{x_i}}{\partial t}
    &=
    \sign \left( \Delta x_i \right)
    \left(
    u \left( \mathbf{x} \right)
    - u \left( \mathbf{x} + \Delta x_i \mathbf{e}_i \right)
    \right)
    \frac{1}{\abs{\Delta x_i}}
    \\
    \frac{m}{V} \frac{\partial v_{x_i}}{\partial t}
    &=
    \left(
    u \left( \mathbf{x} \right)
    - u \left( \mathbf{x} + \Delta x_i \mathbf{e}_i \right)
    \right)
    \frac{\sign \left( \Delta x_i \right)}{\abs {\Delta x_i}}
    \\
    \frac{m}{V} \frac{\partial v_{x_i}}{\partial t}
    &=
    \left(
    u \left( \mathbf{x} \right)
    - u \left( \mathbf{x} + \Delta x_i \mathbf{e}_i \right)
    \right)
    \frac{1}{\sign \left( \Delta x_i \right) \abs{\Delta x_i}}
    \\
    \frac{m}{V} \frac{\partial v_{x_i}}{\partial t}
    &=
    \left(
    u \left( \mathbf{x} \right)
    - u \left( \mathbf{x} + \Delta x_i \mathbf{e}_i \right)
    \right)
    \frac{1}{\Delta x_i}
    \\
    \frac{m}{V} \frac{\partial v_{x_i}}{\partial t}
    &=
    \frac{
    \left(
    u \left( \mathbf{x} \right)
    - u \left( \mathbf{x} + \Delta x_i \mathbf{e}_i \right)
    \right)
    }
    {\Delta x_i}
    \\
    \frac{m}{V} \frac{\partial v_{x_i}}{\partial t}
    &=
    -
    \frac{
    \left(
    u \left( \mathbf{x} + \Delta x_i \mathbf{e}_i \right)
    - u \left( \mathbf{x} \right)
    \right)
    }
    {\Delta x_i}
\end{align*}

Переходя к пределу получаем
\(
\rho \frac{\partial v_{x_i}}{\partial t} = - \frac{\partial u}{\partial x_i}
\)
или
\(
\frac{\partial v_{x_i}}{\partial t}
= - \frac{1}{\rho} \frac{\partial u}{\partial x_i}
\)
.

Таким образом.

\begin{align*}
    \frac{\partial v_x}{\partial t}
    &= - \frac{1}{\rho} \frac{\partial u}{\partial x}
    \\
    \frac{\partial v_y}{\partial t}
    &= - \frac{1}{\rho} \frac{\partial u}{\partial y}
    \\
    \frac{\partial v_z}{\partial t}
    &= - \frac{1}{\rho} \frac{\partial u}{\partial z}
\end{align*}

Из уравнения состояния, при использовании нескольких приближений можно получить
\(
\frac{\partial u}{\partial t}
=
-
\rho
c^2
\left(
\frac{\partial v_x}{\partial x}
+ \frac{\partial v_y}{\partial y}
+ \frac{\partial v_z}{\partial z}
\right)
\).

Из системы уравнений

\begin{align*}
    \frac{\partial v_x}{\partial t}
    &= - \frac{1}{\rho} \frac{\partial u}{\partial x}
    \\
    \frac{\partial v_y}{\partial t}
    &= - \frac{1}{\rho} \frac{\partial u}{\partial y}
    \\
    \frac{\partial v_z}{\partial t}
    &= - \frac{1}{\rho} \frac{\partial u}{\partial z}
    \\
    \frac{\partial u}{\partial t}
    &=
    -
    \rho
    c^2
    \left(
    \frac{\partial v_x}{\partial x}
    + \frac{\partial v_y}{\partial y}
    + \frac{\partial v_z}{\partial z}
    \right)
\end{align*}

можно получить волновое уравнение:

\begin{align*}
    \frac{\partial^2 u}{\partial t^2}
    &= \frac{\partial}{\partial t} \frac{\partial u}{\partial t} \\
    &=
    \frac{\partial}{\partial t}
    \left(
    -
    \rho
    c^2
    \left(
    \frac{\partial v_x}{\partial x}
    + \frac{\partial v_y}{\partial y}
    + \frac{\partial v_z}{\partial z}
    \right)
    \right)
    \\
    &=
    - \rho c^2
    \frac{\partial}{\partial t}
    \left(
    \frac{\partial v_x}{\partial x}
    + \frac{\partial v_y}{\partial y}
    + \frac{\partial v_z}{\partial z}
    \right)
    \\
    &=
    - \rho c^2
    \left(
    \frac{\partial}{\partial t} \frac{\partial v_x}{\partial x}
    + \frac{\partial}{\partial t} \frac{\partial v_y}{\partial y}
    + \frac{\partial}{\partial t} \frac{\partial v_z}{\partial z}
    \right)
    \\
    &=
    - \rho c^2
    \left(
    \frac{\partial}{\partial x} \frac{\partial v_x}{\partial t}
    + \frac{\partial}{\partial y} \frac{\partial v_y}{\partial t}
    + \frac{\partial}{\partial z} \frac{\partial v_z}{\partial t}
    \right)
    \\
    &=
    - \rho c^2
    \left(
    \frac{\partial}{\partial x}
    \left( - \frac{1}{\rho} \frac{\partial u}{\partial x} \right)
    + \frac{\partial}{\partial y}
    \left( - \frac{1}{\rho} \frac{\partial u}{\partial y} \right)
    + \frac{\partial}{\partial z}
    \left( - \frac{1}{\rho} \frac{\partial u}{\partial z} \right)
    \right)
    \\
    &=
    - \rho c^2
    \left(
    - \frac{1}{\rho} \frac{\partial}{\partial x} \frac{\partial u}{\partial x}
    - \frac{1}{\rho} \frac{\partial}{\partial y} \frac{\partial u}{\partial y}
    - \frac{1}{\rho} \frac{\partial}{\partial z} \frac{\partial u}{\partial z}
    \right)
    \\
    &=
    \left( - \rho c^2 \right) \left( - \frac{1}{\rho} \right)
    \left(
    \frac{\partial}{\partial x} \frac{\partial u}{\partial x}
    + \frac{\partial}{\partial y} \frac{\partial u}{\partial y}
    + \frac{\partial}{\partial z} \frac{\partial u}{\partial z}
    \right)
    \\
    &=
    c^2
    \left(
    \frac{\partial^2 u}{\partial x^2}
    + \frac{\partial^2 u}{\partial y^2}
    + \frac{\partial^2 u}{\partial z^2}
    \right)
\end{align*}

Будем решать задачу численно используя метод Finite-Difference Time-Domain.
Для этого необходима дискретизация по временным и пространственным переменным.

Узел, в котором вычисляется давление,
должен быть окружён узлами, в которых вычисляются компоненты вектора скорости.
Расположение узлов отличается
от расположения
в методе Finite-Difference Time-Domain для законов Ампера и Фарадея.

Кроме того, необходимо смещение по временной переменой
- узлы, в которых вычисляется давление, должны быть смещены на полшага
от узлов, в которых вычисляются компоненты вектора скорости.

Будем считать пространственный шаг одинаковым \( h_x = h_y = h_z = h \).

Заменим производные конечными разностями.

\begin{align*}
    u^l \left[ i, j, k \right]
    &\approx u \left( ih, jh, kh, l \Delta_t \right)
    \\
    v_x^{l + \frac{1}{2}} \left[ i, j, k \right]
    & \approx
    v_x
    \left(
    \left( i + \frac{1}{2} \right) h,
    jh,
    kh,
    \left( l + \frac{1}{2} \right) \Delta_t
    \right)
    \\
    v_y^{l + \frac{1}{2}} \left[ i, j, k \right]
    & \approx
    v_y
    \left(
    ih,
    \left( j + \frac{1}{2} \right) h,
    kh,
    \left( l + \frac{1}{2} \right) \Delta_t
    \right)
    \\
    v_z^{l + \frac{1}{2}} \left[ i, j, k \right]
    & \approx
    v_z
    \left(
    ih,
    jh,
    \left( k + \frac{1}{2} \right) h,
    \left( l + \frac{1}{2} \right) \Delta_t
    \right)
    \\
    u^l \left[ i, j, k \right]
    &= 
    u^{l-1} \left[ i, j, k \right]
    -
    \frac{\Delta_t \rho c^2}{h}
    \left( \right.
    \\
    &v_x^{l - \frac{1}{2}} \left[ i, j, k \right]
    - v_x^{l - \frac{1}{2}} \left[ i - 1, j, k \right]
    \\
    &+ v_y^{l - \frac{1}{2}} \left[ i, j, k \right]
    - v_y^{l - \frac{1}{2}} \left[ i, j - 1, k \right]
    \\
    &+ v_z^{l - \frac{1}{2}} \left[ i, j, k \right]
    - v_z^{l - \frac{1}{2}} \left[ i, j, k - 1 \right]
    \left. \right)
    \\
    v_x^{l + \frac{1}{2}} \left[ i, j, k \right]
    &=
    v_x^{l - \frac{1}{2}} \left[ i, j, k \right]
    -
    \frac{\Delta_t}{h \rho}
    \left( u^l \left[ i + 1, j, k \right] - u^l \left[ i, j, k \right] \right)
    \\
    v_y^{l + \frac{1}{2}} \left[ i, j, k \right]
    &=
    v_y^{l - \frac{1}{2}} \left[ i, j, k \right]
    -
    \frac{\Delta_t}{h \rho}
    \left( u^l \left[ i, j + 1, k \right] - u^l \left[ i, j, k \right] \right)
    \\
    v_z^{l + \frac{1}{2}} \left[ i, j, k \right]
    &=
    v_z^{l - \frac{1}{2}} \left[ i, j, k \right]
    -
    \frac{\Delta_t}{h \rho}
    \left( u^l \left[ i, j, k + 1 \right] - u^l \left[ i, j, k \right] \right)
    \\
\end{align*}

Шаг по времени нужно выбирать так, чтобы число Куранта не превосходило единицу.

\begin{align*}
    C = \frac{\sup \left( c \right) \Delta_t}{h} \\
    C \le 1 \\
    \frac{\sup \left( c \right) \Delta_t}{h} \le 1 \\
    \Delta_t \le \frac{h}{\sup \left( c \right)}
\end{align*}

Для простоты в дальнейшем будем рассматривать случай \( n = 2 \):

\begin{align*}
    \frac{\partial v_x}{\partial t}
    &= - \frac{1}{\rho} \frac{\partial u}{\partial x}
    \\
    \frac{\partial v_y}{\partial t}
    &= - \frac{1}{\rho} \frac{\partial u}{\partial y}
    \\
    \frac{\partial u}{\partial t}
    &=
    -
    \rho
    c^2
    \left(
    \frac{\partial v_x}{\partial x} + \frac{\partial v_y}{\partial y}
    \right)
\end{align*}

\begin{align*}
    u^l \left[ i, j \right] & \approx u \left( ih, jh, l \Delta_t \right) \\
    v_x^{l + \frac{1}{2}} \left[ i, j \right]
    & \approx
    v_x
    \left(
    \left( i + \frac{1}{2} \right) h,
    jh,
    \left( l + \frac{1}{2} \right) \Delta_t
    \right)
    \\
    v_y^{l + \frac{1}{2}} \left[ i, j \right]
    & \approx
    v_y
    \left(
    ih,
    \left( j + \frac{1}{2} \right) h,
    \left( l + \frac{1}{2} \right) \Delta_t
    \right)
    \\
    u^l \left[ i, j \right]
    &= 
    u^{l-1} \left[ i, j \right]
    -
    \frac{\Delta_t \rho c^2}{h}
    \left( \right.
    \\
    &v_x^{l - \frac{1}{2}} \left[ i, j \right]
    - v_x^{l - \frac{1}{2}} \left[ i - 1, j \right]
    + v_y^{l - \frac{1}{2}} \left[ i, j \right]
    - v_y^{l - \frac{1}{2}} \left[ i, j - 1 \right]
    \left. \right)
    \\
    v_x^{l + \frac{1}{2}} \left[ i, j \right]
    &=
    v_x^{l - \frac{1}{2}} \left[ i, j \right]
    -
    \frac{\Delta_t}{h \rho}
    \left( u^l \left[ i + 1, j \right] - u^l \left[ i, j \right] \right) \\
    v_y^{l + \frac{1}{2}} \left[ i, j \right]
    &=
    v_y^{l - \frac{1}{2}} \left[ i, j \right]
    -
    \frac{\Delta_t}{h \rho}
    \left( u^l \left[ i, j + 1 \right] - u^l \left[ i, j \right] \right) \\
\end{align*}

Заменим аппроксимацию производной по пространственным переменным
на центральную аппроксимацию 12-того порядка.
Формулы не приводятся по причине их громоздкости.

\begin{figure}
    \caption{Начальные условия}
\noindent\includegraphics[width=\textwidth]{initial}
\end{figure}

\begin{figure}
    \caption{Решение прямой задачи с однородным граничным условием Неймана}
\noindent\includegraphics[width=\textwidth]{forward_neyman_zero_boundary}
\end{figure}
